\documentclass[11pt]{article}

\usepackage[utf8]{inputenc}
\usepackage{amsmath,amssymb,amsthm} % for math symbols/environments
\usepackage{graphicx}               % for including images
\usepackage{hyperref}               % for hyperlinks

\title{Assignment 1 \\\vspace{1in}}


\author{Andrew Ungureanu \\
        \\40283344}
\date{Tuesday, March 4th 2025\\
            SOEN 331}

\begin{document}

\maketitle

\newpage
\section{Problem 1: Propositional Logic (7 pts)}

\subsection*{1.1 Statement by Sophia the robot \,(3 pts)}


\subsection*{1.2 Argument from “Computing Machinery and Intelligence” \,(4 pts)}


\newpage
\section{Problem 2: Predicate Logic (8 pts)}

\subsection*{2.1 Interpreting given formalizations \,(4 pts)}
\begin{itemize}
    \item (a) \(\forall x : \mathbb{R} : \text{number}(x) \to \neg \text{rational}(x)\)
    \item (b) \(\exists x : \mathbb{R} : \text{number}(x) \wedge \text{rational}(x)\)
\end{itemize}

\subsection*{2.2 Formalizing statements \,(4 pts)}
\begin{itemize}
    \item (a) “There are no irrational real numbers.”
    \item (b) “Some real numbers are irrational.”
\end{itemize}

\newpage
\section{Problem 3: Linear Temporal Logic 1 (15 pts)}

\subsection*{3.1 (3 pts)}
% ...

\subsection*{3.2 (3 pts)}
% ...

\subsection*{3.3 (3 pts)}
% ...

\subsection*{3.4 (3 pts)}
% ...

\subsection*{3.5 (3 pts)}
% ...

\newpage
\section{Problem 4: Linear Temporal Logic 2 (15 pts)}

\subsection*{4.1 Visualizing all models of behavior \,(9 pts)}
% The big formula:
%  \( \Box \bigl[\dots\bigr] \)

\subsection*{4.2 Mathematical structures \,(4 pts)}
% ...

\subsection*{4.3 Observations on termination,\newline non-termination, consistency \,(2 pts)}
% ...

\newpage
\section{Problem 5: Unordered Structures (10 pts)}

% 10 sub-questions on sets and membership
% e.g. Is \(\{Kirk\} \in P(\text{Names})\), etc.

\newpage
\section{Problem 6: Ordered Structures (10 pts)}

\subsection*{6.1 Implementing a Queue using two lists \,(4 pts)}
% Using elements \(\langle 1, 2, 3 \rangle\)

\subsection*{6.2 Defining \(\text{Enqueue}\) and \(\text{Dequeue}\) \,(6 pts)}
% ...

\newpage
\section{Problem 7: Binary Relations, Functions, and Orderings (15 pts)}

\subsection*{7.1 Poset proofs and Hasse diagrams}
\begin{itemize}
    \item (a) \(\bigl(\{2,3,4,6,24,36,72\}, /\bigr)\) is a poset \,(2 pts) \\
          Hasse diagram, maximal/minimal elements
    \item (b) \(\bigl(\mathcal{P}\{1,2,3\}, \subseteq\bigr)\) is a poset \,(1 pt) \\
          Hasse diagram, maximal/minimal elements
\end{itemize}

\subsection*{7.2 Analyzing variable \(\texttt{map}\) \,(12 pts)}
% Domain/codomain, partial/total, injectivity, surjectivity, etc.

\newpage
\section{Problem 8: Binary Relations, Functions, and Orderings 2 (10 pts)}

\subsection*{8.1 Is \(\texttt{map2}\) a function, domain/codomain \,(2 pts)}
% ...

\subsection*{8.2 Properties \,(8 pts)}
% injective, surjective, order preserving, isomorphism, etc.

\newpage
\section{Problem 9: Construction Techniques (10 pts)}

\subsection*{9.1 Function \(\texttt{map}(f,\Lambda)\) \,(5 pts)}
\begin{itemize}
    \item (a) Transform into computable function
    \item (b) Define function recursively
    \item (c) Unfold your definition for \(\texttt{map}(f,\langle a,b,c\rangle)\)
\end{itemize}

\subsection*{9.2 Function \(\texttt{insert}(x,\Lambda)\) \,(5 pts)}
\begin{itemize}
    \item (a) Transform into computable function
    \item (b) Define function recursively
    \item (c) Unfold example: \(\texttt{insert}(5,\langle 1,3,5,7\rangle)\)
\end{itemize}

\newpage


\end{document}
