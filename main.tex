\documentclass[11pt]{article}

\usepackage[utf8]{inputenc}
\usepackage{amsmath,amssymb,amsthm} % for math symbols/environments
\usepackage{graphicx}               % for including images
\usepackage{hyperref}               % for hyperlinks

\title{Assignment 1}
\author{Your Full Name \\
        Your Student ID}
\date{\today}

\begin{document}

\maketitle

\begin{abstract}
\noindent
\textbf{Course:} SOEN~331 \\
\textbf{Instructor:} Dr.\,Constantinos Constantinides \\
\textbf{Due Date:} Tuesday, 4 March 2025 (23:59) \\
\textbf{Weight:} 10\% of the overall grade
\end{abstract}

\vspace{1em}

\section*{General information}
% (Optional) Provide any brief, high-level remarks here if needed.

\section*{Introduction and ground rules}
% Summarize or restate any ground rules from the assignment.
% For instance: 
% 1) This is an individual assessment. 
% 2) Must be prepared in LaTeX. 
% 3) Follow submission instructions strictly.

\newpage
\section{Problem 1: Propositional Logic (7 pts)}

\subsection*{1.1 Statement by Sophia the robot (3 pts)}
% Placeholder for your analysis and formal logic representation.

\subsection*{1.2 Argument from “Computing Machinery and Intelligence” (4 pts)}
% Placeholder for your analysis of validity.

\newpage
\section{Problem 2: Predicate Logic (8 pts)}

\subsection*{2.1 Interpreting given formalizations (4 pts)}
% (a) ...
% (b) ...

\subsection*{2.2 Formalizing statements (4 pts)}
% (a) ...
% (b) ...

\newpage
\section{Problem 3: Linear Temporal Logic 1 (15 pts)}

\subsection*{3.1 Requirement “if exactly one of ϕ, ψ becomes invariant…” (3 pts)}
% ...

\subsection*{3.2 Requirement “if ϕ and ψ differ at time = i…” (3 pts)}
% ...

\subsection*{3.3 Describing and visualizing ¬(¬ϕ ∨ ¬ψ) → ◇²τ ∧ ○² ◇ π (3 pts)}
% ...

\subsection*{3.4 Describing and visualizing ○² ◇²τ ∧ (χ ∧ ○ψ) → ○²² ◇ (β R α) (3 pts)}
% ...

\subsection*{3.5 Describing and visualizing ○²(α ⊕ β) → ○◇ (τ U κ) (3 pts)}
% ...

\newpage
\section{Problem 4: Linear Temporal Logic 2 (15 pts)}

\subsection*{4.1 Visualizing all models of behavior (9 pts)}
% ...

\subsection*{4.2 Mathematical structures (4 pts)}
% ...

\subsection*{4.3 Observations on termination, non-termination, consistency (2 pts)}
% ...

\newpage
\section{Problem 5: Unordered Structures (10 pts)}
% 10 sub-questions. Summarize each question carefully.

\newpage
\section{Problem 6: Ordered Structures (10 pts)}

\subsection*{6.1 Implementing a Queue using two lists (4 pts)}
% ...

\subsection*{6.2 Defining Enqueue and Dequeue operations (6 pts)}
% ...

\newpage
\section{Problem 7: Binary Relations, Functions, and Orderings (15 pts)}

\subsection*{7.1 Poset proofs and Hasse diagrams (3 pts total)}
% (2 pts) poset on {2,3,4,6,24,36,72} 
% (1 pt) poset on P\{1,2,3}

\subsection*{7.2 Analyzing variable map (12 pts total)}
% (a) Domain, codomain, total vs partial
% (b) Properties: injective, surjective, etc.

\newpage
\section{Problem 8: Binary Relations, Functions, and Orderings 2 (10 pts)}

% Similar format as Problem 7 but focusing on map2
\subsection*{8.1 Is map2 a function, domain, codomain (2 pts)}
% ...

\subsection*{8.2 Properties analysis (8 pts)}
% ...

\newpage
\section{Problem 9: Construction Techniques (10 pts)}

\subsection*{9.1 Function \texttt{map}(f, \(\Lambda\)) (5 pts)}
% (a) Transform into computable function
% (b) Define recursively
% (c) Unfold example

\subsection*{9.2 Function \texttt{insert}(x,\(\Lambda\)) (5 pts)}
% (a) Transform into computable function
% (b) Define recursively
% (c) Unfold example

\newpage
\section*{What to submit}
% Remind yourself (and the grader) of the submission instructions:
% 1) .tex file named as per your name/id
% 2) A folder /images with all figures
% 3) Zip and submit on EAS

\end{document}s